
\documentclass[11pt,english]{article}
\usepackage[T1]{fontenc}
\usepackage[latin9]{inputenc}
\usepackage{geometry}
\geometry{verbose,tmargin=1in,bmargin=1in,lmargin=1in,rmargin=1in}
\synctex=-1
\usepackage{babel}
\usepackage{amsthm}
\usepackage{amsmath}
\usepackage{setspace}
\onehalfspacing


\title{}
\author{Zhentao Shi}

\begin{document}
A very simple example.

The nonparametric estimator in Xiao's paper.
(2009. p.83, above Theorem 1)
\[
\hat{\beta}(z) = \left[ \frac{1}{nh} \sum_{t=1}^T x_t x_{t}^{\prime} K \left( \frac{z_t-z}{h} \right) \right]^{-1}
 \frac{1}{nh} \sum_{t=1}^T x_t y_t K \left( \frac{z_t - z}{h} \right) 
\]
Given the data, please code up this estimator.

Naive idea: two loops. For each fixed $z$, do a loop from $t=1$ to $T$ for summation. Then go over each $z = z_i$. 

two points of improvement

Vecotization: $\sum_{t=1}^T x_t x_{t}^{\prime} K \left( \frac{z_t-z}{h} \right)$ can be written in matrix form as $X' K(z) X$.

plyr: save the book keeping in loops.

Another example of loop. Cluster variance.


\end{document}
